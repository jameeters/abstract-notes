\documentclass[11pt, letterpaper]{article}
\usepackage{mathtools}
\usepackage{amsfonts}
\usepackage{amsmath}
\usepackage[makeroom]{cancel}
\usepackage{amssymb}


\newcommand{\ord}[1]{\lvert{#1}\rvert}
% \newcommand{\rm}[1]{\mathrm{{#1}}}
\newcommand{\stab}{\mathrm{stab}}
\newcommand{\orb}{\mathrm{orb}}
\newcommand{\ordiv}[2]{\ord{{#1}} \mid \ord{{#2}}}
\newcommand{\genby}[1]{\langle {#1} \rangle}
% \newcommand{\ker}{\mathrm{ker}}
\newcommand{\inv}{^{-1}}

\begin{document}
\title{Abstract Notes}
\author{James McFeeters}
\date{\today}
\maketitle

\section{Basic Group Properties}
	\subsection{Definition of a Group}
		\paragraph{Binary relation:}
			If $G$ is a set, $*$ is a binary relation on $G$ if $a*b \in G$ for all $a, b \in G$.

		\paragraph{Group}
			Let $G$ be a set.
			$G$ is a group if there exists a binary relation on $G$ that satisfies the following properties for all $a, b, c \in G$:
			\begin{enumerate}
				\item $a (bc) = (ab) c$ (associative) 
				\item There exists an identity element $e \in G$.
				\item Every element has an inverse
			\end{enumerate}

		\paragraph{Identity:}
			An element $e$ is an identity of $G$ if $ae = ae = a$ for all $a \in G$.

		\paragraph{Inverse:}
			Let $G$ be a group and let $a \in G$.
			By the inverse of $a$ we mean an element $b \in G$ such that $ab = ba = e$.

		\paragraph{Abelian group:}
			If $ab = ba$ for all $a, b \in G$, then the group is called abelian.
			(Abelian groups are those for which the commutative property holds.)

		\paragraph{Order:}
			The order of $G$, denoted $\ord{G}$, refers to the number of elements in $G$. \\
			The order of an element $a$, denoted $\ord{a}$, is the smallest positive integer $n$ such that $a^n = e$.

		\paragraph{Cyclic groups:}
			A cyclic group is one which can be generated by just one element of the group using the operation of the group. 
			The group generated by an element $a$ is denoted $\genby{a}$


	\subsection{Basic Theorems / Properties}
		\paragraph{Prop. 1:}
			The identity is unique.

		\paragraph{Prop. 2:}
			Suppose $a, b, x \in G$ where $G$ is a group.
			If $ax = bx$ then $a = b$, and if $xa = xb$ then $a = b$.
			(The right and left cancellation laws hold.)

		\paragraph{Prop. 3:}
			The inverse of any element $a \in G$ is unique.

		\paragraph{Socks-shoes property:}
			For group elements $a$ and $b$, $(ab)\inv = b\inv a\inv$.
			Because you put on your socks, then your shoes: $ab$.
			Then to reverse the operation, you take off your shoes, then your socks: $b\inv a\inv $



\section{(.*)morphisms}
	\subsection{Definitions}
		\paragraph{Isomorphism:}
			$G$ and $\bar G$ are said to be isomorphic if there exists a one-to-one and onto function $\phi: G \mapsto \bar G$ such that $\phi(ab) = \phi(a) \phi(b)$. 

		\paragraph{Automorphism:}
			An isomorphism $\phi: G \mapsto G$ is called an automorphism of $G$.

	\subsection{Theorems}
		\paragraph{Proposition:}
			Suppose $\phi: G \mapsto \bar G$ is an isomorphism.
			Then:
			\begin{enumerate}
				\item $\phi(e) = \bar e$ 
				\item $\phi(a\inv) =(\phi(a))\inv $ 
				\item $\phi(a^s) = (\phi(a))^s $ for any integer $s$.
			\end{enumerate}
			In re 3: We write $a^{-k}$ where $k$ is a positive integer to denote $\underbrace{a\inv a\inv \dots a\inv}_{k \text{ times}}$

\section{Groups and Subgroups}
	\subsection{Definitions}
		\paragraph{Subgroup:}
			Let $G$ be a group. 
			If a subset $H$ of $G$ is a group under the operation of $G$, then $H$ is a subgroup of $G$. \\
			Let $H$ be a subgroup of $G$.
			If $H \neq G$ and $H \neq \{e\}$, then $H$ is called a \emph{proper subgroup} of $G$.

	\subsection{Theorems}
		\paragraph{}
			Any subgroup of a cyclic group is itself cyclic.

\section{Cosets:}
	\subsection{Definition}
		\paragraph{Left and right cosets:}
			Let $G$ be a group and let $H$ be a subgroup of $G$.
			We define left and right cosets of $H$ in $G$ as follows: \\
			Let $a \in G$, then 
			\[ \text{the set ~} aH = \{ ah \mid h \in H \} \text{ ~is called a left coset of $H$ in $G$} \]
			and
			\[ \text{the set ~} Ha = \{ ha \mid h \in H \} \text{ ~is called a right coset of $H$ in $G$} \]
			The element $a$ is called the coset representative of $aH$ or $Ha$

		\paragraph{Index:}
			The index of a subgroup $H$ in $G$ id the number of distinct left cosets of $H$ in $G$, denoted $\ord{G:H}$

	\subsection{Theorems and Properties}

		\begin{enumerate}
			\item $a \in aH$ 
			\item $aH = H \iff a \in H $
			\item $(ab)H = a(bH)$ and $ H(ab) = (Ha)b $
			\item $aH = bH \iff a \in bH$
			\item $ aH = bH $ or $ aH \cap bH = \varnothing $
			\item $ aH = bH \iff a\inv b \in H $
			\item $ \ord{aH} = \ord{bH} = \ord{H} $
			\item $ aH = Ha \iff H = aHa\inv $
			\item $aH$ is a subgroup of $G \iff a \in H$
		\end{enumerate}

		\paragraph{Lagrange's Theorem: }
			If $H$ is a subgroup of a finite group $G$, then $\ordiv{H}{G}$. \\
			Also, the number of distinct left/right cosets of $H$ in $G$ is $\frac{\ord{G}}{\ord{H}}$

		\paragraph{Corollaries:}
			\begin{enumerate}
				\item $\ord{G:H} = \frac{\ord{G}}{\ord{H}}$
				\item $ \ordiv{a}{G} $ for all $a \in G$
				\item Groups of prime order are cyclic.
				\item $a^{\ord{G}} = e$ (since the order of the group must be a multiple of the order of the element)
			\end{enumerate}

		\paragraph{Theorem: }
			Suppose $G$ is a group and $H$ and $K$ are subgroups of $G$.\\
			$\ord{HK} = \displaystyle\frac{\ord{H}\ord{K}}{\ord{H \cap K}}$ \\
			Where $HK = \{ hk \mid h \in H \text{ and } k \in K \}$








\clearpage
\hrule

\section{Disorganized Notes}

% Things below here are not yet organized 


\paragraph{Lagrange's Theorem: }
	If $H$ is a subgroup of a finite group $G$, then $\ordiv{H}{G}$.

\subsubsection{Order of an Element}
	The order of an element $a$, denoted $\ord{a}$, is the smallest positive integer $n$ such that $a^n = e$.

\paragraph{}
	$\genby{a} = $ the group generated by $a$.

\paragraph{Prop. 1: }
	If $a \ in G$, then $\ordiv{a}{G}$

\paragraph{Prop. 2: }
	Take $\mathbb{Z}_p^*$ where $p$ is prime.
	Then $\ord{\mathbb{Z}_p^*} = p-1$




\subsubsection{Stabilizer}
	The stabilizer of $i$ in $G$ is the set of elements of $G$ that fix $i$. \\
	Denoted $\stab_G (i)d$.

		\paragraph{Prop: }
			$\stab_G(i)$ is a subgroup of $G$.

\subsubsection{Orbit}
	The orbit of $i$ in $G$ is the set of all $g(i)$ where $g \in G$.
	Denoted $\orb_G(i)$

\paragraph{Theorem: }
	$\ord{G} = \ord{\stab_G(i)} \ord{\orb_G(i)}$


\subsubsection{Normal Subgroup}
	Let $G$ be a group.
	A subgroup $N$ is called a normal subgroup of $G$ if $gN = Ng$ for all $g \in G$.
	This does not mean that $gn = ng$ for all $n \in N$.
	It does mean that $gn = n_1 g$ for some $n_1 \in N.$

\subsubsection{Index of a Subgroup}
	Let $H$ be a subgroup of $G$.
	Then the index of $H$ in $G$ is the number of distinct left (or right) cosets of $H$ in $G$. \\
	$\text{index} = \frac{\ord{G}}{\ord{H}}$



\subsubsection{External Direct Product}
	Let $G_1$ and $G_2$ be two groups.
	The external direct product is defined as \\
	$G_1 \oplus G_2 = \{ (x, y) | x \in G_1, y \in G_2 \}$.\\
	And the product is defined by $(x, y)(x_1, y_1) = (xx_1, yy_1)$.\\
	Observe that if $G_1$ and $G_2$ are abelian, then $G_1 \oplus G_2$ is abelian.

\subsubsection{Internal Direct Product}
	Suppose $H$ and $K$ are normal subgroups of $G$.
	$G$ is called the internal direct product of $H$ and $K$ if $G = HK$ and $ H \cap K = \{ e \}$

\paragraph{Theorem: }
	Let $G$ be an abelian group of order $n$.
	Suppose $p$ is a prime and $p | n$.
	Then there exists an element $g \in G$ that is of order $p$.

\subsubsection{Center of $G$:}
	$Z(G) = \{ g \in G | gh = hg \text{ for all } h \in G\}$ \\

	\paragraph{Prop: }
		$Z(G)$ is a normal subgroup of $G$.

\paragraph{Theorem: }
	$Z(G)$ is a normal subgroup of $G$.

\subsubsection{Homeomorphism}
	A homeomorphism is a mapping
	\begin{align*}
		\phi: &G \mapsto G' \\
		&e \mapsto e '\\
	\end{align*}
	if $\phi(ab) = \phi(a) \phi(b)$ for all $a, b \in G$


\subsubsection{Kernel}
	The kernel of $\phi$ is the set of all elements in $G$ that are mapped to $e'$.
	Denoted $\ker \phi = \{ g \mid g \in G, \phi(g) = e' \} $

\paragraph{Prop: }
$\ker \phi$ is a normal subgroup of $G$.

\paragraph{Theorem: }
	Suppose $\phi: G \mapsto G'$ is an onto homeomorphism.
	Then $G/\ker\phi $

\end{document}

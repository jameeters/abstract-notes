\documentclass[11pt, twoside, letterpaper]{article}
\usepackage{mathtools}
\usepackage{amsfonts}
\usepackage{amsmath}
\usepackage[makeroom]{cancel}
% \usepackage[margin = 0.75 in]{geometry}
% \usepackage[cm]{fullpage}
% \usepackage{extsizes}

\newcommand{\ord}[1]{|{#1}|}
% \newcommand{\rm}[1]{\mathrm{{#1}}}
\newcommand{\stab}{\mathrm{stab}}
\newcommand{\orb}{\mathrm{orb}}
\newcommand{\ordiv}[2]{\ord{{#1}} \mid \ord{{#2}}}
\newcommand{\genby}[1]{\langle {#1} \rangle}
% \newcommand{\ker}{\mathrm{ker}}

\begin{document}
\title{Abstract Notes}
\author{James McFeeters}
\date{\today}
\maketitle


\paragraph{Lagrange's Theorem: }
	If $H$ is a subgroup of a finite group $G$, then $\ordiv{H}{G}$.

\subsubsection{Order of an Element}
	The order of an element $a$, denoted $\ord{a}$, is the smallest positive integer $n$ such that $a^n = e$.

\paragraph{}
	$\genby{a} = $ the group generated by $a$.

\paragraph{Prop. 1: }
	If $a \ in G$, then $\ordiv{a}{G}$

\paragraph{Prop. 2: }
	Take $\mathbb{Z}_p^*$ where $p$ is prime.
	Then $\ord{\mathbb{Z}_p^*} = p-1$
\paragraph{}
	Suppose $G$ is a group and $H$ and $K$ are subgorups of $G$. \\
	Then $HK = \{ hk \mid h \in H \text{ and } k \in K \}$

\paragraph{Theorem: }
	$\ord{HK} = \displaystyle\frac{\ord{H}\ord{K}}{\ord{H \cap K}}$

\subsubsection{Stabilizer}
	The stabilizer of $i$ in $G$ is the set of elements of $G$ that fix $i$. \\
	Denoted $\stab_G (i)d$.

		\paragraph{Prop: }
			$\stab_G(i)$ is a subgroup of $G$.

\subsubsection{Orbit}
	The orbit of $i$ in $G$ is the set of all $g(i)$ where $g \in G$.
	Denoted $\orb_G(i)$

\paragraph{Theorem: }
	$\ord{G} = \ord{\stab_G(i)} \ord{\orb_G(i)}$


\subsubsection{Normal Subgroup}
	Let $G$ be a group.
	A subgroup $N$ is called a normal subgroup of $G$ if $gN = Ng$ for all $g \in G$.
	This does not mean that $gn = ng$ for all $n \in N$.
	It does mean that $gn = n_1 g$ for some $n_1 \in N.$

\subsubsection{Index of a Subgroup}
	Let $H$ be a subgroup of $G$.
	Then the index of $H$ in $G$ is the number of distinct left (or right) cosets of $H$ in $G$. \\
	$\text{index} = \frac{\ord{G}}{\ord{H}}$



\subsubsection{External Direct Product}
	Let $G_1$ and $G_2$ be two groups.
	The external direct product is defined as \\
	$G_1 \oplus G_2 = \{ (x, y) | x \in G_1, y \in G_2 \}$.\\
	And the product is defined by $(x, y)(x_1, y_1) = (xx_1, yy_1)$.\\
	Observe that if $G_1$ and $G_2$ are abelian, then $G_1 \oplus G_2$ is abelian.

\subsubsection{Internal Direct Product}
	Suppose $H$ and $K$ are normal subgroups of $G$.
	$G$ is called the internal direct product of $H$ and $K$ if $G = HK$ and $ H \cap K = \{ e \}$

\paragraph{Theorem: }
	Let $G$ be an abelian group of order $n$.
	Suppose $p$ is a prime and $p | n$.
	Then there exists an element $g \in G$ that is of order $p$.

\subsubsection{Center of $G$:}
 	$Z(G) = \{ g \in G | gh = hg \text{ for all } h \in G\}$ \\

	\paragraph{Prop: }
		$Z(G)$ is a normal subgroup of $G$.

\paragraph{Theorem: }
	$Z(G)$ is a normal subgroup of $G$.

\subsubsection{Homeomorphism}
	A homeomorphism is a mapping
	\begin{align*}
		\phi: &G \mapsto G' \\
		&e \mapsto e '\\
	\end{align*}
	if $\phi(ab) = \phi(a) \phi(b)$ for all $a, b \in G$


\subsubsection{Kernel}
	The kernel of $\phi$ is the set of all elements in $G$ that are mapped to $e'$.
	Denoted $\ker \phi = \{ g \mid g \in G, \phi(g) = e' \} $

\paragraph{Prop: }
$\ker \phi$ is a normal subgroup of $G$.

\paragraph{Theorem: }
	Suppose $\phi: G \mapsto G'$ is an onto homeomorphism.
	Then $G/\ker\phi $

\end{document}
